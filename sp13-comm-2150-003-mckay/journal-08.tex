\documentclass[12pt,letterpaper]{article}
\usepackage{mla}
\usepackage{wrapfig}

% overwrite double spacing
\linespread{1}

\begin{document}
\begin{mla}{Paul}{English}{COMM 1010-003}{Shirene
    McKay}{\today}{\textbf{Reflection Journal - Chapter 8 - Transitions}}
    
\section{Area Competency}

Interpersonal interest 

\section{Goal}

To improve communication with other individuals by working on social interest and connectedness in conversation and general interest.

\section{Tactic}

To find situations where intercultural transitions occur. If found to probe, and see if I can learn more about the feelings and type of cultural transition that has occurred.

\section{Journal Entries}

\subsection{Entry One}

I have a coworker who is a mathematician. Seriously, he received his PhD at the UofU, and taught classes there. He's very talented, has some great (still available online) course material, and has published several papers relating stochastic processes to biology found in the brain. He studied brains mathematically. I really wanted to know why he would change careers, why he would enter the private sector. He simply told me that he got tired of working on problems that no one wanted, or was interested in. That's strange. There is definitely an interest there. I'm sure it's very small, but still, it's potentially very useful work. There is no doubt that the world outside of academia moves differently. It can be very easy to find problems that many people want , and the reward for doing so is often great. It is quite the transition.

\subsection{Entry Two}

I spent the weekend working in the "Transition Tent" as a volunteer for a relay run event in Southern Utah. At first it didn't relate to cultural transitions at all. It was just making sure one runner gets his/her digitally tracked number to the next so they can start the next leg of the run. It was absurd. Hundreds of people, all ages, all races, all levels of fitness running in high-altitude for long-distances, and having to coordinate and rely on the performance of teammates. The cultural transition that took place was temporary. Inside the tent everyone is either anticipating the run, or just returning from their run and usually quite exhausted. They pass they're tag on to the next runner as though it's a weight. Maybe helping them put on the tag, and then encouraging their partner on the next leg.

The excitement was infectious, even though it was 3AM (the competition ran from afternoon, all night, to afternoon the next day). Everyone helped everyone, as for most the goal was not to win the race, simply to finish it. I suppose there is a transition there of running cultural, with those who are new to running culture. There was an infusion of camping culture, and party culture, amongst the individual cultures of each team. Many of the teams had themes, were part of related groups, or had particular style.

It's a bit off from my original tactic, but simply participating in the event was quite the example of cultural transition. Also I tried to pace run on a shorter trail with a teammate (70+ years of age), and he left me in the dust about a third of the way into the trail. I really need to get in good cardio shape\dots

\end{mla}
\end{document}

