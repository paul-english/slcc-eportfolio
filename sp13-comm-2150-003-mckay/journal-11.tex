\documentclass[12pt,letterpaper]{article}
\usepackage{mla}
\usepackage{wrapfig}

% overwrite double spacing
\linespread{1}

\begin{document}
\begin{mla}{Paul}{English}{COMM 1010-003}{Shirene
    McKay}{\today}{\textbf{Reflection Journal - Chapter 11 - Conflict}}
    
\section{Area Competency}

Interpersonal interest 

\section{Goal}

To improve communication with other individuals by working on social interest and connectedness in conversation and general interest.

\section{Tactic}

I hope to recognize apparent, but ambiguous points of conflict in my life currently, and analyze my reasoning and positions on them. Should I be more considerate? Should I focus on improving communication? Should I seek further action? Should I admit defeat? Most importantly, in-line with my goal, is this an issue of not taking personal interest in the other parties involved in this conflict?

\section{Journal Entries}

\subsection{Entry One}

So we've got this great project in a programming class of mine that involves building a relatively simple web app with a team. I tend to have no problem finishing CS work, so I have lead the team in implementation and development. We're building a toll that has predictive capabilities using (mostly) statistical analysis. It was apparent at the beginning that one member would slack regularly, and the other would be in over his head, being unfamiliar with the concepts, ideas, and algorithms used to build this tool. In no way did I wish to complicate the project by attacking a hard problem, but they offered no simpler replacement. I believe there is an ambiguous and silent conflict occuring in this situation. I can tell that the slacker would like to contribute, but makes no effort to manipulate the code, respond to updates on the project, or even ask questions. The other guy, who is at least trying, wants to make a difference, but doesn't want to change ideas I've implemented, or shoot down any complex ideas. I'll have no problems completing the assignment, but as it stands I've committed easily ninety percent of the work, and he's done maybe ten. I know he doesn't want to fail, but has not attempted to implement things that have been discussed in class, or shape the idea at all. I don't believe that this is due to a lack of communication, but it's sensible to presume I'm very biased in that assessment. How do we properly save face as teammates, all three of us, knowing well aware our respective contributions to the project? I know I don't plan to judge either character only by the work they've put into the project. I just wonder what I should suggest when reviewing about their contribution to the professor for grading. I know that our slacker focuses primarily on work, at least it would seem that way. Our eager, but slightly incapable, companion would hope he gets counted for the time he's spent where he could. And of course, myself, the apparent know it all has been the pack mule.

\subsection{Entry Two}

I have this Econ teacher who is certainly bright, and capable, but seems a bit scatter brained and doesn't really have too much control over the class time. The course content has been interesting, and the required work has been fine, however the final project, a timeline of historical economic events presented throughout the reading, but the execution of the project has been bad. Our team is good on this one, five members, three who contribute, and two who don't. We've gotten our work done, early at that, but have found trouble using the professors prescribed method of templating the timeline. We used a Google spreadsheet, a tool that is collaborative and "real-time", meaning changes are propagated from machine to machine, as though we're literally typing at the same computer. We used a freely available open-source tool that presents the spreadsheet data in a very clean timeline format. Our professor would like us to use Microsoft Word, and use a proprietary and commercial (read as buggy and hard to use) plugin to produce expandable/collapsible boxes for storing information. Though it's possible to make interactive "documents" using Microsoft Word, it's a bit like using a hammer to paint an impressionist scene. It's the wrong tool for the job, and though it's available to everyone, is not accessible, bordering on restrictive. Every team is having difficulties and finding the process to be painful. This is a case where the method of project execution inhibits and undermines the purpose of the project, which is to share a very comprehensive timeline of economic events in the history of our country with our classmates, and to improve our overall understanding of the material we've just studied. So we're all at conflict with this teachers methods, and the teacher is well aware of it. Has the teacher done a good thing or a bad thing with this project? Is it a mistake that is not being fully admitted too? I informed the teacher early of what I could see to be a potential problem by email. I offered my solution, and tried to share my reasoning in using it. Later as the deadline has approached and all teams have had difficulties, I've gotten some acknowledgement in the ideas and opinions I shared initially.

\end{mla}
\end{document}

