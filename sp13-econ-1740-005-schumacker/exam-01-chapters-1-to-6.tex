\documentclass[12pt,letterpaper]{article}

\usepackage{mla}
\usepackage{wrapfig}
\usepackage{setspace}

\begin{document}
\begin{mla}{Paul}{English}{ECON 1740-005}{Heather
    Schumacker}{\today}{\textbf{Exam I Chapters 1-6}}

\singlespacing
\section{Discuss the importance of and implication of trade with
  regards to the colonies prior to 1776}
\doublespacing

The seemingly explosive growth in the early colonial period of the
United States was largely a product of inter-continental trade with
Britain, and other surrounding territories and overseas areas.
Everything from the discovery and inhabitation of these lands, to the
long-term establishment of colonies here, and eventual division and
independence from English rule can in some way relate to the trade and
markets that grew around these locations.

It Started with early exploration and European discovery of the
America's, lead by Portugal, who was the leading pioneer in Atlantic
growth. Their exploration was driven by trade and colonization of new
territories including areas in North Africa, Madeira, the Canary
Islands, among other locations. Columbus, upon reaching the Americas,
helped to start an upward trend of growth and immigration devoted to
trade and export of many of the natural resources found in these
areas.

Portugal was taken over by Spain, and their settlements in America
lacked solid long-lasting foundation. However, other countries had
also began to settle and control this area, continuing growth and
immigration. These settlements and colonies had to seek out ways to
both live and subside, which meant establishing farms, fishing, and
manufacture of ships. Enabling them to begin producing their own
goods, and trading with the countries that helped to establish them in
this area.

Cash crops, including tobacco, rice, indigo, and other grains,
flourished across all of the Southern regions of North America and
even into some of the Middle areas during the colonial era. Along the
north furs, forest goods, lumber, and ships were popular and useful in
the inter-continental trade. They had unique values on the markets,
since the same good couldn't be produced as plentifully in other areas
at the time.

As England took control of the area that would later become the United
States, exports subsisted largely of tobacco, bread and flour, rice,
fish, wheat, indigo, corn, pine boards, staves and headings, and horses
as pointed out by the text book. Most of the imports came from England
and were manufactured goods, which the colonists could not adequately
create in their new settlements. This diversity of imports and exports
is a good example of the value of the trade relationship which had
grown around the new territories. England profited from the unique and
plentiful natural resources that were found and cultivated in these
new areas, while the colonists still relied on the goods that they had
grown accustomed to from before their immigration. 

Since much of the trade revolved around the cash crops, slavery became
popular in Southern regions in order to deal with many of the
unfavorable conditions and back breaking labor that was typically
required to harvest large plantations. 

Much of this trade continued as the colonies grew, and eventually
became independent from the English throne. It also helped to
influence many other things including the establishment of common
currency, establishment of credit, and shifting and changing
distribution of wealth.

If it wasn't for the thriving trade that took place in the first
centuries of colonial existence growth and immigration wouldn't have
been prominent.

%- lack of gold and mining in the early decades
%- cash crops helped drive market production and trade
%- land and natural resources helped determine development

%- taxes on trade; tobacco, navigation acts 
%- though there were taxes placed on many things they caused
%practically no disruption in established trade patterns

%- tobacco, rice, and indigo provided half the value of the top 10
%exports

%- imports from england were primarily manufactured goods

%- top 10 exports; tobacco, bread and flour, rice, fish, wheat, indigo,
%corn, pine boards, staves and headings, horses

%- primarily traded with england. but also the west indies, southern
%europe, and some with africa

%- establishment of currency

%- trade deficits with england
%- mercantilist measures were setup to favor trade balances for england
%- required the US to make up for their trade deficits with trade in
%the west indies, other overseas areas

%- growth of credit

%- shigting and adjusting of wealth in the colonies


\pagebreak

\singlespacing
\section{What was the mind set of those coming to settle the American
  colonies (include who came, why did they come, how did they come)
  and what was the mind set of England with regards to the colonies in
  America?}
\doublespacing

The early American colonies were known as the New World, which largely
helps us to decipher the mind set of the people who chose to
immigrate and settle these areas. It was a location that offered new
opportunities, new challenges and standards of living that couldn't be
had anywhere else they had previously been. They chose to find a way across
the Atlantic through rather difficult and expensive means with a hope
that they could take on some of the challenges and receive the
opportunities that lay in store. We know that many were religious, and seeking
to escape tyranny. Many were poor or of common descent, and there were
also many who were highly regarded and wealthy. There were many who
we're seeking out the chance to be a land-owner, which was regarded as
a symbol of high status in the land-locked areas they were from. 

Portugal and Spain led the colonization, for trade and exploratory
reasons, however their settlements didn't last, only to be replaced by
long-term settlements established primarily by England. As previously
mentioned, these immigrants sought out land, religious freedom, and
opportunities they otherwise wouldn't have had in their home country. 

Skilled laborers, artisans, and craftsmen were highly valued in the
New World and tended to be more successful, therefore they tended to
be the type of person that would travel to get here. Families did
immigrate though were less likely to do so because of the cost and
difficulty of the journey. 

Since traveling by ocean-liner was an expensive affair, many of the
immigrants bartered and took loans out to pay for the passage. This
included loans with ship captains, who they could pay back upon
arrival, as well as indentured servitude contracts that required them
to work off their debts after arriving. This practice eventually
subsided as family members in the new colonies began to afford paying
for their family members and relatives to make the trip without having
to borrow or contract their way into a ticket for the new world. 

During this time of colonial growth, England followed largely
mercantilist systems of thinking. This largely meant that they sought
power and wealth for their own state. This lead to taxes, and strict
systems that favored trade for themselves, and strictly regulating the
economic life of the American colonies. The English wanted to keep
artisans, and skilled workers at home so they could continue their
domestic industry. 

England controlled much of the trade that took place in the new
colonies accepting most of the exports, and providing most of their
imports. The measures that England took made sure that colonial trade
resulted in a favorable outcome for them, and a deficit for the
colonists. They established the navigation acts, restricted the export
of their coins, and continually exploited the economic situation of
the colonies which they had established. 

%new world
%- new opportunities and challenges

%- portugal were some of the first
%- spain also an early colonizer
%- both lacked what was needed to create long-term settlements

%- holland france and england start to show up
%- with france and england becoming the chief competitors
%- british colonies maintained steady and persistant growth

%- commoners were desirous to own land
%- commoners had the opportunity to reach a higher standard of living
%- also religious reasons

%- indenture contracts and expensive passage
%- by 19th century indentured contracts largely disappeared

%- tradesman, lacking families
%- artisans and laborers

\pagebreak

\singlespacing
\section{What were the major sources of productivity for the colonists
and why did different regions choose to produce the goods and services
that they did?}
\doublespacing

%- cash crops
%- tobacco, rice

%- land and natural resources determined the path of development and
%economic activity in the various colonies
%- from new hampshire to georgia agriculture was the chief occupation,
%what industrial and commercial activity they had revolved around
%materials extracted from the land

%- areas with soil and climate unfavorable for crops, they turned to
%fishing and trapping, as well as the production of ships, and other
%forest products

%- farming farming farming

The early colonists primarily farmed, fished, and lived off natural
resources that were found in the new lands they now occupied. The
differing climate and conditions of each area helped to differentiate
and establish each colony with specialties and strengths which they
could leverage to help themselves grow. This included plantations and
cash crops in the southern regions, farms that harvested wheat, flour,
and grains in the middle colonies, and fishing and port towns in the
northern areas. 

%southern colonies
%- fertile new land, revolved around a few staples
%- tobacco; required long growing season and fertile soil, could be
%cultivated in small areas or partially cleared fields
%- rice; required periodic flooding and draining
%- indigo; later introduced
%- other commodities; deerskins, naval stores, hay, animal products,
%indian corn, wheat, other grains, livestock
%- plantations
%- crops for export to britain
%-- large plantations, single crops, nearby port cities


The Southern areas were filled with fertile land and had favorable
climates that allowed for long growing seasons. tobacco became the
largest staple, filling up entire plantations, and sometimes requiring
the labor of many slave teams to harvest and manage. Tobacco also
could be grown in small areas if necessary, and only needed partially
cleared fields to thrive. Rice was also a staple of the South, though
it requires periodic flooding and draining, a technique which was
achievable near swamplands and rivers, where water could be diverted
during certain seasons. They also produced Indigo, a plant introduced
from India that is useful for creating blue dyes. Additionally other
commodities that were produced include deerskins, naval stores, hay,
animal products, indian corn, wheat, other grains, and livestock. Many
of these crops which were produced in the Southern areas were exported
to England, so there were also port cities that arose. Primarily the
South had the largest farms, and produced the most grown goods. 

%middle colonies
%- fertile and tillable
%- trees
%- wheat and flour; also corn, rye, oats, barley
%- farms, smaller than plantations
%- crops for colonies in the north
%-- larger farms, urban centers home to merchants and artisans

In the Middle colonies the land was fertile and tillable, however
farms tended to be smaller than the large plantations of the South.
They primarily produced wheat and flour, as well as corn, rye, oats,
and barley. Their crops mainly benefited the Northern colonies, and
not as much emphasis was put on the export of their goods to Britian.
They had large farms, and urban centers that were home to merchants,
artisans, and craftsmen. 

%new england
%-- fishing villages, port cities, small towns with modest family farms
%- poor soils, uneven terrain, and severe climate
%- not a good place for commercial farming
%- indian corn, could be produced almost anywhere and had satisfactory
%yeild even on poor land
%- other farming was for family use
%- some cattle and sheep raising
%- net importer of food
%- shipping and fishing were major economic activities
%- cod, whaling

In New England, and the Northern colonies, the land was filled with
poor soils, uneven terrain, and severe climate. It was a bad location
for any type of commercial farming, so the farms here were mostly for
family use. These farms were usually able to grow indian corn, and
raise some cattle and sheep. They imported most of their food from
other colonies. Since farming wasn't profitable in these areas they
turned to fishing and shipping. They fished cod, and even whale. They
had some of the largest port cities, and they also worked at
shipbuilding. Shipbuilding was the most extensive form of manufacture
in the New World, since it could be done with the plentiful lumbar and
forest materials, and iron ore that was found in these areas. 

Manufacturing in general across all the colonies wasn't a huge
producer, aside from shipbuilding in the North. The colonies did
however manufacture some things like clothing, food, woodwork, and
specialty craftsmen industries. Most of their manufactured goods were
imported from England where they had plentiful factories, mills, and
other necessary tools and resources to do the job. 

Each colony primarily worked with the resources that were plentiful
around them. They all shifted into production that was based on the strengths of
their areas, and by what was marketable at the time. The South
wouldn't have grown as much tobacco as they did if there wasn't a
demand for it, or England didn't want to have a source of tobacco that
wasn't Spanish. The North and Middle colonies weren't able to sustain
large plantations, otherwise they might have tried.

%further inland
%- more ore and raw iron was found

%- manufacturing
%- clothing, food, woodwark, craftsman industries
%- shipbuilding in the north
\end{mla}
\end{document}

