\documentclass[12pt,letterpaper]{article}
\usepackage{mla}
\usepackage{wrapfig}
\usepackage{setspace}

\begin{document}

\begin{mla}{Paul}{English}{HUMA 1100-012}{Derek
    Bitter}{\today}    
    {\textbf{Reflection Journal - That Damned Fence}}

\section*{Why did we let the internment of Japanese American citizens happen?}

It's sad to think that at about the same time the Nazi party was interning Jewish people, and all walks of life that didn't fit their standards into concentration camps, that we were doing something similar. In some ways you can't compare the two acts, we certainly didn't push the issue into genocide, but we did offend many rights which these people had. What caused the issues, why would we allow it?

Fear plays a large role in the action. We were at war, and we couldn't trust the size of our borders, or the immigrant nationalities of many of the people now living hear. We were attacked, and didn't want to get attacked again. So we did something, certainly not the best thing, but like a lot of political work some action often takes the place of the right action, simply because group think wants us to respond quickly to events that occur.

The mindset that we take on when at war helped to make it easy as well. We became very polarized, we saw them as the enemy, whether they were living here or living in Japan. Their blood and heritage made them evil to us, and our country decided that we had to do something to prevent that evil from hurting us.

There are other things that led to this action I'm sure, but I think the last few lines of the poem help sum up the issue succintly, ``We all love life, and our country best, Our misfortune to be here in the west, To keep us penned behind that DAMNED FENCE, Is someone's notion of NATIONAL DEFENCE!''



%%%%
%* How were the Japanese internment camps able to happen?
%
%  - anger towards the japanese
%    - blindness
%  - overgeneralizations
%    - easy
%    - mental
%      - order
%      - stability
%  - fear
%  - safety
%  - propaganda
%  - innocence
%  - american
%  - easy
%  - uneducated reaction
%  - mythic reality
%  - blame
%    - scapegoat
%  - emotion
%  - suspicion
%  - group think / mob mentality
%  - false connections/assumptions
%  - self appointed judges


\end{mla}
\end{document}

