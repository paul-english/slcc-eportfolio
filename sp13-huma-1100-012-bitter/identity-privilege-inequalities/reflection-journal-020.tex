\documentclass[12pt,letterpaper]{article}
\usepackage{mla}
\usepackage{wrapfig}
\usepackage{setspace}

\begin{document}

\begin{mla}{Paul}{English}{HUMA 1100-012}{Derek
    Bitter}{\today}    
    {\textbf{Reflection Journal - Getting Out the Vote}}

\section*{What did it mean for women at this time, not having the ability to vote?}

It's weird to think of what it would be like to not allow a major portion of a population to vote, and still try to claim democracy. Without a vote, in political affairs at least, you have no voice. No ability to support, protest, or improve things which go on in our local and federal governments. The reading points out examples including protection for women and their children, nourishment, education, water sanitization, prices for basic necessities, and even wages are all political matters that a woman should have an equal say in. There isn't a political matter out there that doesn't ultimately affect women in some way. Everything we choose to do in this country has an affect on every citizen, whether positive or negative. How then, given that women had no voice to begin with, did they challenge their right, and persuade the passage of the Sixteenth Amendment?

I'm sure it wasn't easy, we have similar issues today, but unlike then, most of the issues today allow all involved parties to not only voice their opinion to the public, they have the right to vote. Equal marriage rights for gay and lesbian couples is an example of this. They can both protest, and vote. Women during this election could only speak out to the men who could vote. Surely this factor postponed the inevitable passage of this amendment, who knows how much sooner it could have been done if women had more power to change it.


\end{mla}
\end{document}

