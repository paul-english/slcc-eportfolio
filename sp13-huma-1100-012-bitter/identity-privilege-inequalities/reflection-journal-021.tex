\documentclass[12pt,letterpaper]{article}
\usepackage{mla}
\usepackage{wrapfig}
\usepackage{setspace}

\begin{document}

\begin{mla}{Paul}{English}{HUMA 1100-012}{Derek
    Bitter}{\today}    
    {\textbf{Reflection Journal - The Story of an Hour}}

\section*{Why would a woman be happy to hear the news of her husbands death?}

In this short story a woman, Mrs. Mallard, is observed after hearing news of the untimely death of her husband. She cries, and sobs in public, retreating to her room. Once alone she finds the feeling relief overwhelming her, she even tries to deny it at first, but is shown to be very grateful. She is happy of the fact that she will be able to live for herself, instead of perhaps pandering to the relationship, or lack thereof, she has with her husband. Upon regaining some composure, she rejoins her sister, and goes downstairs. The door opens to reveal her husband safe and sound, a site which surprises Mrs. Mallard to death. Mrs Mallard is described as having a heart condition at the beginning of the story, and we can only assume the shear surprise of seeing someone again after feeling the relief she did presumably caused her heart to fail.

What kind of life was she living that would cause her to feel a relief over her husbands death? I'm sure that no authentic relationship would have such an issue. One obvious reason would be the formality, and arranged nature of marriages in the time. Women roles in society were to get married, it was the mans responsibility to work and provide. The women pleased the man. At least that's what this story probably had in mind. This girls identity while married was consumed by her husband in every way, and finally upon imagining the death, she was able to truly feel what her individualist identity could provide her. It may have been something she never felt before, and as soon as it was felt, it was quickly snatched away from her.

It's certainly not nice to feel joy over the death of someone, but I don't think Mr. Mallard, if he knew, should take it personally. He probably wasn't too bad of a person, maybe just not the person for Mrs Mallard. All the same, the nature of marriage in this time meant that she had no choice, and could only be a good wife if she was submissive and agreeable. It's no wonder she shocked herself to sleep realizing her situation at her husbands return.

\end{mla}
\end{document}

