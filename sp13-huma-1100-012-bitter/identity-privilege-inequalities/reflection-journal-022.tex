\documentclass[12pt,letterpaper]{article}
\usepackage{mla}
\usepackage{wrapfig}
\usepackage{setspace}

\begin{document}

\begin{mla}{Paul}{English}{HUMA 1100-012}{Derek
    Bitter}{\today}    
    {\textbf{Reflection Journal - Left-Handed Commencement Address}}

\section*{What is left-handed about this commencement speech?}

I saw this reading in the list for this section, and immediately wondered what it was about. I'm left-handed, I honestly thought I might get a chance to read something that fought for the equal rights of left-handed people everywhere. Some kind of preached awareness for the struggles we face, maybe empathy and consoling for always having to wash our hands after handwriting any kind of essay with a pencil or most pens. I was gearing myself up to disagree in fact, I can't say as a left-handed person that I've ever been severely disadvantaged. Instead I was greeted with an address that made no mention to a left-hand dominant physical trait, but instead offered a straightforward viewpoint of feminism. Left-handed was purely meant as an opinion coming from the other side, the kind of commencement speech that goes against the grain.

The author, Ursula K. Le Guin, retorts about the unfavorable position of the woman in a mans society. How in a man's world, it is often a man's voice that gets heard, and that is something which she feels she must talk out against. That the unique viewpoints, and disadvantaged position offered only to women should be brought to light. That we'll all be better for it, man and woman. She knows that a woman will not, and probably should not, seek to imitate what a man does, but instead embrace the their own living. To her, the identity of women is important, so important that she speaks to it directly in her commencement, making little effort to address the men of the audience. Not that we all can't learn from what she has to say. When the author titled her speech left-handed, what she really meant was closer to the definition of dubious and of doubtful sincerity, though I'd like to imagine she's looking out for us left-handed writers as well.


\end{mla}
\end{document}

