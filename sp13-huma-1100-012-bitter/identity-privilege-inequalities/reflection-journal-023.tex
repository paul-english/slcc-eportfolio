\documentclass[12pt,letterpaper]{article}
\usepackage{mla}
\usepackage{wrapfig}
\usepackage{setspace}

\begin{document}

\begin{mla}{Paul}{English}{HUMA 1100-012}{Derek
    Bitter}{\today}    
    {\textbf{Reflection Journal - Black Boy}}

\section*{How did the author's initial success plague him in both good and bad ways?}

The Author, also protagonist of the story, tells us about his first success as a writer, and goes on to describe the life he sought to live after that point. He began, from sheer idleness, he describes, by writing a short simple story. He submitted it to a local newspaper, and was lucky enough to have it published as a series of three parts. He then goes on to describe how he got by, and eventually how he went about procuring books from the library and reading them, moving from author to author and novel to novel. He self-educated himself using the works of others, and yet his success and drive though ultimately liberating, also held consequences. He says describing about his situation in life, ``I no longer felt that the world about me was hostile, killing; I knew it. A million times I asked myself what I could do to save myself, and there were no answers. I seemed forever condemned, ringed by walls.'' A strange outlook, though true in some senses. 

A black man, in the south, shortly after the civil war, was only marginally in a better situation than before and during the war. Tensions were still strained, and just because the law said these people were now free, didn't stop many from disagreeing, and being violent about it. They were in a disadvantaged position, and the Author became more and more wise to this fact as he read. What was before only a simple understanding, had grown into a more complex wisdom afforded through the viewpoints he absorbed in the books he read. He pushed against the grain in a time when others in his situation didn't think it was right to do so. By becoming more and more aware of his situation he was able to recognize how down-trodden his people had been. Though he described himself as more trapped than ever, he really was becoming more privileged. There is no doubt that being aware of an issue is at least somewhat better than being affected by the issue and completely unaware. However when unaware, we can at least be comforted by our ignorance, blissful. He was no longer ignorant and unaware, he was now burdened with responsibility over the things which he had learned.



\end{mla}
\end{document}

