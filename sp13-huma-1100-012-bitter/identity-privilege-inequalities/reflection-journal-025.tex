\documentclass[12pt,letterpaper]{article}
\usepackage{mla}
\usepackage{wrapfig}
\usepackage{setspace}

\begin{document}

\begin{mla}{Paul}{English}{HUMA 1100-012}{Derek
    Bitter}{May 1, 2013}    
    {\textbf{Reflection Journal - The Absolutely True Diary of a Part-Time Indian}}

\section*{Why did the author feel the need to fight over an insult?}

The story we read presents a view of an outsider, anxious and uneasy in a new surrounding. He's a Indian in a school filled with white privileged kids. He sees himself as self-ostracized from the culture he grew up in, an outsider who can no longer go back to the school he had left. He describes how quickly he is now seen as an outsider in his new school. He's stuck now with no possible hope of being a part of any group. He's very aware that he's alone in this new school, and that he represents all of his culture, and all of the people he so eagerly left. When faced with an insult what was he supposed to do? The kid who threw the insult at him, was quite brash, maybe he didn't understand what gravity his joke had to Junior, our author. 

Junior takes a moment midway through the story to describe all of the unwritten rules which he had come to understand, growing up on the reservation. It's all about the family, any insult whatever it may be to family, parents, mothers, brothers was to be dealt with physically in a fight, often with the adequate repercussions of another fight of some sort. Though the Author found himself scared, outnumbered, and under-sized in this group of peers, to him the only logical reaction was to punch the guy. To be the first to throw a punch, so that he didn't get knocked out having not thrown one at all. To him this was just the way it was, these kids were so insulting because they knew they could fight and win, at least that's what he thought initially. Turns out they were only there to bark at him, and didn't believe there would be any escalation. Surprised they ran off leaving the author without a sense for what to do next. Had he won? Was he supposed to fight? Was there some new rules he was supposed to play by?

The Author uses the title part-time indian probably to describe how he assimilates his new culture, but falls back to his upbringing on the reservation in scenes like this. He had no idea these kids didn't want to fight him, he only knew what would have happened if the same thing occurred amongst a group of Native American's where he was from. Fearing for his safety he acted quickly to save face, which had the consequences of leaving him unharmed and definitely the victor of the fight he'd just been in. Was it right to punch the kid in the face? He did insult the kid, two races, and the buffalo species, so in a few ways it was probably deserving. Though I doubt the author will be so haste to throw a punch in the future knowing that this new group of kids tends to be a little more mean verbally, and doesn't seem to mean for it to get physical. 

\end{mla}
\end{document}

