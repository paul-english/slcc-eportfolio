\documentclass[12pt,letterpaper]{article}
\usepackage{mla}
\usepackage{wrapfig}

\begin{document}
\begin{mla}{Paul}{English}{HUMA 1100-012}{Derek Bitter}{\today}{\textbf{Reflection Journal \#7 - American Indian}}

\section{How should we treat the sacred things of others?}

%%%%%%%%%%%%%%%%%%%%%%%%%%%%%%%%%%%%%%%%%%%%%%%%%%
% How should we treat & view the sacred things of others?

% - What about our own?

We cannot really change others, and though we may not always share
similar beliefs, if we are to respect another people at all, we must
also respect their customs, their traditions, and anything they
consider sacred. It would be wrong to try and convince another culture
that a practice of theirs isn't sacred. Even the most rational,
logical thinkers find certain things to be sacred. It's a part of all
human beings, and something that helps us to experience a full range
of emotion. So why on earth would you deny someone, or a group this
right purposefully and expect others to allow it for yourself.

With that said, it can be easy to make mistakes and not recognize
something that is sacred to another. I've always found the terrain of
the United States to be awe inspiring. I can understand the sacredness
that is present in American Indian nations for these spaces that we
tend to take for granted. I can also see how easy it is to forget the
sacredness and perhaps be disrespectful, but as a point of respect, we
should always pay attention to the things others find sacred.

\end{mla}
\end{document}

