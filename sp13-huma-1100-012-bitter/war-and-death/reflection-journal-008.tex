\documentclass[12pt,letterpaper]{article}
\usepackage{mla}
\usepackage{wrapfig}

\begin{document}
\begin{mla}{Paul}{English}{HUMA 1100-012}{Derek
    Bitter}{\today}{\textbf{Reflection Journal \#8 - Henry V}}

\section*{How responsible are military leaders for the death of soldiers?}

Differing wars, differing fights all change the situation and responsibility that leaders have for those who must follow. In the case that a state is compelled to war, maybe in defense of their territory or their allies, the leaders are usually well supported, and the responsibility falls onto the people who fight. It's easy to be ``right'' when defending your country, it's harder when you take an offensive stance. In ``Henry V'' we read about a king who is seeking out legitimacy and approval from his subjects by going to war. Seemingly because it's what kings do. From the short excerpt that we read, it's easy to get a feeling of distrust or low morale amongst the soldiers. Henry spends some time amongst his soldiers in cover. Not letting on to the fact that he is the king, his conversation with the others helps to give him insight into the position he's in, and how the people feel about this war he's placed them in. At one point a character named Bates helps to sum up the feel, ``For we know enough, if we know we are the kings subjects: if his cause be wrong, our obedience to the king wipes the of it out of us.''

Henry's role as king for his people makes him much more responsible for the war that they are in. As Bates said, the kings cause is what makes them innocent for any evil that might be take place in the upcoming battle. This is mostly true, but every man still has free will, and the ability to choose their side. It seems that these people, may seek to place blame on their king, but they would equally receive the glory if their cause turned into a righteous one.

Though we can spend time analyzing and dissecting what people would be morally right or wrong in a situation like this, we might never know for sure. History tends to favor the victors, which by default would more often make the side that loses feel more responsible, it would make the people expect more responsibility of their leaders for the lives of the people lost and the hardships faced. The side that won can more easily rationalize atrocities and deaths by claiming glory over defeat.



\end{mla}
\end{document}

