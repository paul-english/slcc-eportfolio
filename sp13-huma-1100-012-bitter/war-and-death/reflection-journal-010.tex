\documentclass[12pt,letterpaper]{article}
\usepackage{mla}
\usepackage{wrapfig}
\usepackage{setspace}

\begin{document}

\begin{mla}{Paul}{English}{HUMA 1100-012}{Derek
    Bitter}{\today}    
    {\textbf{Reflection Journal \#10 - Speech to the Third Army: The Unabridged Version}}

\section*{Do we need motivation in war?}

Is everyone motivated when they sign up for battle? Do they know what they're getting into? Do they all join willingly? Why would we throw ourselves into danger when there is high probability that we may not last more than a few minutes? George Patton doesn't exactly seem like a relatable character, but he sure seems like he fit into a war setting. Whatever his background, whatever his ideals, the one thing it seems he was able to convey was motivation in the face of dangerous situations. In his speech to the Third Army he was direct, honest, focused on winning, and quite the patriot. He didn't censor his speech, not that the military is a place where one would need to restrain violent language, all the same he painted a graphic image. He says in his speech,

\begin{blocks}
``War is a bloody, killing business. You've got to spill their blood,
or they will spill yours. Rip them up the belly. Shoot them in the
gus. When shells are hitting all around you and you wipe the dirt off
your face and realize that instead of dirt it's the blood and guts of
what once was your best friend beside you, you'll know what to do!''
\end{blocks}

Few people want to put themselves in danger, but many, I hope, would do it to protect their friends, or their family, and most certainly a country that promotes their beliefs. Morale for the U.S. Army was probably high during World War II, given the manner we entered. Still, we all need pep talks every once and a while, and if life is on the line, as it was in this situation, you might as well make it a memorable one.

Maybe in any situation like this, whether favorable to war or not, you need the kind of imagery and strong direction which Patton inflicted upon his army. Knowing that in situations like this you have strong leadership, is going to positively influence you and those around you no matter what. Knowing that your peers are in agreement helps you move forward. War is depressing, but it happens, and when it does happen it's useful to know that you can take your side and valiantly fight knowing that you're a strong team. I certainly appreciate the speech, though I still can't relate to the situation fully, and I respect the man.

\end{mla}
\end{document}

