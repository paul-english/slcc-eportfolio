\documentclass[12pt,letterpaper]{article}
\usepackage{mla}
\usepackage{wrapfig}
\usepackage{setspace}

\begin{document}

\begin{mla}{Paul}{English}{HUMA 1100-012}{Derek
    Bitter}{\today}    
    {\textbf{Reflection Journal \#13 - Three Deaths: A Tale}}

\section*{Do we always respect the dead or dying?}

Sometimes we are respectful and reverent of those around us who are
dying, or those who are already dead. Sometimes we think we're being
respectful, though at most we can only sympathize. No one living can
fully empathize with the situation that everyone will eventually find
themselves going through. We can only speculate at what our time will
be like in the end. Will we be content and happy with the lives we
have lived. Will we be bitter and stagnant in our views, only able to
place blame upon others for where our lives have lead us. Will we even
see it coming or be aware of it when it happens, we can't all die as
wise old sages. Some of us may not even want to make it to that point.

I'm sure we'd like to think that as we age we grow more and more at
peace with our inevitable fate, but I would guess that for most we
reminisce of times past. Most people, no matter how fulfilling a life
they've lead seem to in some way long for the times which have past,
in which they were strong, quick-witted, or capable and happy.

In this weaving story by Leo Tolstoy, we travel through the last days
and hours of two people and one tree. Trying to understand what it's
like to be in the late stages of life, and how those who are younger
than us, and now taking care of us react. How they emotionally
separate themselves from the dying, and move on with living quickly.
In the case of the tree, we could anthropomorphize what it's like to
die. What the other trees feel now that the one is gone. But it
wouldn't make sense, we can only look at it in the most objective
sense. Here was a tree that grew for many years in this one spot, and
one day it was cut down.


\end{mla}
\end{document}

