\documentclass[12pt,letterpaper]{article}
\usepackage{amsmath}

\title{Chem 1010-009 Homework}
\date{\today}
\author{Paul English}

\begin{document}

\maketitle

\begin{enumerate}

\setcounter{enumi}{4} % 5
\item 

Will the power of 10 have a \textit{positive} or a \textit{negative} exponent when each of the following numbers is rewritten in standard scientific notation?

\begin{enumerate}
\item 42,751 - positive \\
\item 1253 - positive \\
\item 0.002045 - negative \\
\item 0.1089 - negative
\end{enumerate}

\setcounter{enumi}{28} % 29
\item 

If you were to measure the width of this page using a ruler, and you used the ruler to the limits of precision permitted by the scale on the ruler, the last digit you would write down for the measurement would be \textit{uncertain} no matter how careful you were. Explain. \\

\begin{itemize}
\item Some degree of uncertainty will always exist when making observed measurements, usually due to visual estimate.
\end{itemize}

\pagebreak

\setcounter{enumi}{40} % 41
\item 

Consider the calculation indicated below: \\
\[\frac{2.21 \times 0.072333 \times 0.15}{4.995}\] \\
Explain why the answer to this calculation should be reported to only two significant digits. \\

\begin{itemize}
\item The item with the least amount of significant digits represents the highest precision we can estimate. Using any more than 2 significant digits wouldn't take into account the uncertainty of $2.21$ or $0.15$.
\end{itemize}

\setcounter{enumi}{46} % 47
\item 

Evaluate each of the following mathematical expressions, and express the answer to the correct number of significant digits. \\

\begin{enumerate}
\item $44.2124 + 0.81 + 7.335 = 52.38$ \\
\item $9.7789 + 3.3315 - 2.21 = 10.90$ \\
\item $0.8891 + 0.225 + 4.14 = 5.25$ \\
\item $(7.223 + 9.14 + 3.7795)/3.1 = 6.50$
\end{enumerate}

\pagebreak

\setcounter{enumi}{58} % 59
\item 

Perform each of the following conversions, being sure to set up the appropriate conversion factor in each case. \\

\begin{enumerate}
\item $12.5 \text{ in.} \times \frac{2.54 \text{ cm}}{1 \text{ in.}} = 31.8 \text{ cm}$\\
\item $12.5 \text{ cm} \times \frac{1 \text{ in.}}{2.54 \text{ cm}} = 4.92 \text{ inches}$\\
\item $2513 \text{ ft} \times \frac{1 \text{ mi}}{5280\text{ ft}} = 0.4759 \text{ mi}$\\
\item $4.53 \text{ ft} \times \frac{1 \text{ mi}}{5280\text{ ft}} \times \frac{1760\text{ yd}}{1\text{ mi}}
\times \frac{1\text{ m}}{1.094\text{ yd}} = 1.38 \text{ meters} $\\
\item $6.52 \text{ min} \times \frac{60 \text{s}}{1 \text{min}} = 391 \text{ seconds} $\\
\item $52.3 \text{ cm} \times \frac{1\text{ m}}{100\text{ cm}} = 0.523 \text{ meters} $\\
\item $4.21 \text{ m} \times \frac{1.094\text{ yd}}{1\text{ m}}= 4.61 \text{ yards} $\\
\item $8.02 \text{ oz} \times \frac{1\text{ lb}}{16\text{ oz}} = 0.501 \text{ pounds}$
\end{enumerate}

\setcounter{enumi}{72} % 73
\item 

Make the following temperature conversions: \\

\begin{enumerate}
\item 44.2 $^\circ$C = 314.2 K \\
\item 891 K =  618 $^\circ$C\\
\item -20 $^\circ$C = 253 K \\
\item 273.1 K = 0.1 $^\circ$C
\end{enumerate}

\pagebreak

\setcounter{enumi}{86} % 87
\item 

For the masses and volumes indicated, calculate the density in grams per cubic centimeter. \\

\begin{enumerate}

\item mass = $452.1 \, g$; volume = $292 \, cm^3$ \\
density = $1.55\,g/cm^3$ \\

\item mass = $0.14 \, lb$; volume = $125 \, mL$ \\
density = $\frac{63.5\,g}{0.0125\,cm^3} = 5030 \,g/cm^3$ \\

\item mass = $1.01 \, kg$; volume = $1000 \, cm^3$ \\
density = $\frac{1010\,g}{1000\,cm^3} = 1 \,g/cm^3$ \\

\item mass = $225 \, mg$; volume = $2.51 \, mL$ \\
density = $\frac{0.225\,g}{0.000251\,cm^3} = 896 \,g/cm^3$

\end{enumerate}


\end{enumerate}

\begin{itemize}

\item \textit{Extra credit:} The prefix name for $10^{-4}$ is decimilli. \\
http://www.mathnstuff.com/math/spoken/here/2class/110/milli/metric.htm
\end{itemize}

\end{document}